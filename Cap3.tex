% Introdução

\documentclass[_ArquivoPrincipal.tex]{subfiles}

\begin{document}
	
\chapter{Metodologia e cronograma}


\section{Metodologia}

\begin{itemize}
	\item Estudar dinâmica das estruturas computacional para problemas com grandes deslocamentos;
	
	\item Estudar técnicas de detecção de contato e auto-contato e técnicas de imposição de condição de não penetração;
	
	\item Estudar dinâmica dos fluidos computacional para escoamentos incomprassíveis com contornos móveis;
	
	\item Estudar técnicas de acoplamento fluido-estrutura particionadas e monolíticas;
	
	\item Desenvolver código para dinâmica não linear de sólidos bidimensionais em formulação posicional;
	
	\item Implementar modelo de contato estrutural do tipo nó-a-superfície;
	
	\item Desenvolver e implementar modelo de contato do tipo superfície-a-superfície;
	
	\item Implementar alterações no programa de mecânica dos fluidos baseado no PFEM em formulação posicional para tornar possível o acoplamento;
	
	\item Acoplar os programas de sólido e de fluido, utilizando o método $\alpha$-\textit{shape} para permitir contato entre estruturas;

\end{itemize}


\section{Cronograma}

De acordo com o desenvolvimento da metodologia, o cronograma das atividades a serem desenvolvidas obedece às seguintes etapas:
\begin{enumerate}[label=\Roman*.]
	\item\label{Rev} Revisão bibliográfica sobre o Método dos Elementos Finitos de Partículas (MEFP), métodos de acoplamento fluido-estrutura e modelos numéricos de interação fluido-estrutura com presença de contato;
	\item\label{prog} Desenvolvimento de programa para análise da interação fluido-estrutura com contato, utilizando a abordagem do MEFP;
	\item\label{model} Desenvolvimento de um modelo \dots
	\item\label{Qual} Qualificação;
	\item\label{valid} Validação dos resultados e modelagem de exemplos;
	\item\label{red} Redação da dissertação e elaboração artigos.	
\end{enumerate}

O cronograma segue conforme a seguinte tabela:
\begin{table}[H]
	\fontsize{10}{14}\selectfont
	\centering	
	
	\definecolor{ativ}{rgb}{0.4,0.5,0.7}
	\begin{tabular}{c|cc|cccccccccccc|cc}
		\hline
		   \multirow{2}{*}{Item} & \multicolumn{2}{c|}{2019} & \multicolumn{12}{c|}{2020} & \multicolumn{2}{c}{2021} \\ \cline{2-17}
		    & Nov & Dez & Jan & Fev & Mar & Abr & Mai & Jun & Jul & Ago & Set & Out & Nov & Dez & Jan & Fev \\ \hline
		 \ref{Rev} & \cellcolor{ativ} & \cellcolor{ativ} & \cellcolor{ativ} & \cellcolor{ativ}& \cellcolor{ativ} & \cellcolor{ativ} & \cellcolor{ativ} & \cellcolor{ativ} & \cellcolor{ativ} & \cellcolor{ativ} & \cellcolor{ativ} & \cellcolor{ativ} & & & & \\ \hline
		 \ref{prog} & \cellcolor{ativ} & \cellcolor{ativ} & \cellcolor{ativ} & \cellcolor{ativ} & & & & & & & & & & & & \\ \hline
		 \ref{model} & & & & & \cellcolor{ativ} & \cellcolor{ativ} & \cellcolor{ativ} & \cellcolor{ativ} & \cellcolor{ativ} & & & & & \\ \hline
		 \ref{Qual} & & & & & & & & & & \cellcolor{ativ} & & & & & & \\ \hline
		 \ref{valid} & & & & & & & & & & & \cellcolor{ativ} & \cellcolor{ativ} & \cellcolor{ativ} & \cellcolor{ativ} & \cellcolor{ativ} & \cellcolor{ativ} \\ \hline
		 \ref{red} & \cellcolor{ativ} & \cellcolor{ativ} & \cellcolor{ativ} & \cellcolor{ativ} & \cellcolor{ativ} & \cellcolor{ativ} & \cellcolor{ativ} & \cellcolor{ativ} & \cellcolor{ativ} & \cellcolor{ativ} & \cellcolor{ativ} & \cellcolor{ativ} & \cellcolor{ativ} & \cellcolor{ativ} & \cellcolor{ativ} & \cellcolor{ativ} \\ \hline
	\end{tabular}
	\caption{Cronograma de atividades}
\end{table}


	
\end{document}
	
	
